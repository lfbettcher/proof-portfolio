\documentclass[12pt]{amsart}
\usepackage{graphicx}
\usepackage{amsfonts}
\usepackage{amssymb,amsmath,latexsym,mathrsfs,amsfonts,amsthm}
\usepackage{eucal}
\usepackage{amscd}
\usepackage{xypic}
\usepackage{mdwlist}
\xyoption{all}
\setlength\parskip{\medskipamount}
\setlength\parindent{0pt}
\pagestyle{empty}
\setlength\parskip{\medskipamount}
\setlength\parindent{0pt}
\setlength{\topmargin}{-.5in}
\setlength{\headheight}{0in}
\setlength{\headsep}{0in}
\setlength{\footskip}{0in}
\setlength{\evensidemargin}{0in}
\setlength{\oddsidemargin}{0in}
\setlength{\textheight}{9.5in}
\setlength{\textwidth}{6.5in}
\setlength{\parindent}{0in}
%----------------------------------------------------------------
\newtheorem{thm}{Theorem}
\newtheorem{cor}[thm]{Corollary}
\newtheorem{lem}[thm]{Lemma}
\newtheorem{prop}[thm]{Proposition}
\newtheorem{exerc}[thm]{Exercise}
\theoremstyle{definition}
\newtheorem{defn}[thm]{Definition}
\theoremstyle{remark}
\newtheorem{rem}[thm]{Remark}
% MATH -----------------------------------------------------------
\newcommand{\nats}{\mathbb N}
\newcommand{\ints}{\mathbb Z}
\newcommand{\rats}{\mathbb Q}
\newcommand{\reals}{\mathbb R}
\newcommand{\complex}{\mathbb C}
\newcommand{\field}{\mathbb F}
\newcommand{\powerset}{\mathscr P}
%----------------------------------------------------------------

\pagestyle{plain}

\begin{document}
\thispagestyle{plain}

\begin{large}{\bf Math 231 \hfill{} Proof Portfolio\hfill{} Name: \underline{Minjia Bettcher}}\end{large}

Please write your proofs neatly in the space provided.  Remember that searching the internet and discussing these questions with anyone, except your instructor or fellow students in this class through the general discussion board, is considered academic dishonesty.\\

\begin{enumerate}
\item The unique positive real number solution to the equation $x^5+x=10$ is irrational.\\

Hints:  You may use the following without justification. \\ \begin{itemize}

\item $\forall n\in\nats$, $\forall a_0,a_1,\ldots, a_n\in\reals$, the poly. $p(x)=a_n x^n +a_{n-1} x^{n-1}+\cdots +a_1 x+a_0$ satisfies $\forall a,b,y\in\reals$ if $y$ is in between $p(a)$ and $p(b)$ then for some $c$ in between $a$ and $b$, $y=p(c)$.\\

\item  Any positive rational number can be represented by a fraction $\frac{p}{q}$ such that $p,q\in\nats$ and the greatest common divisor of $p,q$ is $1$.\\

\item  The function $f(x)=x^5+x$ is an increasing function for $x>0$.

\end{itemize}
\suspend{enumerate}

\bigskip
\bigskip

\emph{Proof}: Consider $x=1$, $(1)^5+(1)=2<10$, and consider $x=2$, $(2)^5+(2)=34>10$.
Since $f(x)=x^5+x$ is an increasing function for $x>0$, there exists a unique positive real number solution $x$ to the equation $x^5+x=10$ such that $x$ must satisfy $1<x<2$.
Suppose that the solution $x$ to $x^5+x=10$ is rational. Then $x$ can be represented by a fraction $p/q$ such that $p,q\in\nats$ and the greatest common divisor of $p,q$ is $1$.
Substitution into the equation yields $\frac{p^5}{q^5}+\frac{p}{q}=10$, $p^5+pq^4=10q^5$, $p(p^4+q^4)=10q^5$, $pr=10q^5$ where r is the integer $p^4+q^4$, so p divides $10q^5$.
Since $p$ and $q$ do not share any prime divisors, $p$ must divide $10$ so the possible values for $p$ are $1, 2, 5$, and $10$.\\
In the case $p=1$, $1<\frac{1}{q}<2$, $\frac{1}{2}<q<1$. There are no integers in $(\frac{1}{2},1)$.\\
\\
In the case $p=2$, $1<\frac{2}{q}<2$, $1<q<2$. Again, there are no integers in $(1,2)$.\\
\\
In the case $p=5$, $1<\frac{5}{q}<2$, $2.5<q<5$. The integers in $(2.5, 5)$ are $3$ and $4$.
\medskip
If $q=3$, $(\frac{5}{3})^5+(\frac{5}{3})=\frac{3530}{243}\neq10$ and if $q=4$, $(\frac{5}{4})^5+(\frac{5}{4})=\frac{4405}{1024}\neq10$, so $q$ cannot be $3$ or $4$.\\
\\
In the case $p=10$, $1<\frac{10}{q}<2$, $5<q<10$. The integers in $(5,10)$ are $6, 7, 8$, and $9$, but $q$ cannot be $6$ or $8$ since it would share the common divisor 2 with $p$, so that leaves $7$ and $9$.
If $q=7$, $(\frac{10}{7})^5+(\frac{10}{7})=\frac{124010}{16807}\neq10$ and if $q=9$, $(\frac{10}{9})^5+(\frac{10}{9})=\frac{165610}{59049}\neq10$, so $q$ cannot be $7$ or $9$.

Therefore there are no integers $p$ and $q$ for $\frac{p^5}{q^5}+\frac{p}{q}=10$, so $x$ is not rational.
That completes the proof$\;_{\square}$\\

\newpage

\resume{enumerate}
\item  For all sets $S,T$, $S\subseteq T$ if and only if $\powerset{(S)}\subseteq \powerset{(T)}$.
\suspend{enumerate}

\bigskip
\bigskip

\emph{Proof}: Let $S,T$ be sets. First let's prove the forward direction: if $S\subseteq T$ then $\powerset{(S)}\subseteq \powerset{(T)}$.
Assume $S\subseteq T$. Let $X$ be an arbitrary element of $\powerset{(S)}$, then $X\subseteq S$. Since  $S\subseteq T$, it follows that $X\subseteq T$.
By the definition of a powerset, $\powerset{(T)}$ is the set of all subsets of $T$, and because $X\subseteq T$ then $X\in \powerset{(T)}$. 
Since $X$ was arbitrary, it follows that all elements of $\powerset{(S)}$ are in $\powerset{(T)}$, so $\powerset{(S)}\subseteq \powerset{(T)}$.

Next let's prove the reverse direction: if $\powerset{(S)}\subseteq \powerset{(T)}$ then $S\subseteq T$.
Assume $\powerset{(S)}\subseteq \powerset{(T)}$. 
Since $S\in \powerset{(S)}$ and $\powerset{(S)}\subseteq \powerset{(T)}$, it follows that $S\in \powerset{(T)}$.
Since S is an element of both $\powerset{(S)}$ and $\powerset{(T)}$, and $\powerset{(S)}\subseteq \powerset{(T)}$, then $S\subseteq T$.

Now that we have proved both the forward and reverse directions, we have proved that for all sets $S,T$, $S\subseteq T$ if and only if $\powerset{(S)}\subseteq \powerset{(T)}$$\;_{\square}$\\

\newpage

\resume{enumerate}

\item For all integers $n\geq 0$, $5^{5n+1}+4^{5n+2}+3^{5n}$ is a multiple of $11$.

\suspend{enumerate}

\bigskip
\bigskip

\emph{Proof}: Let's prove this by induction. First we must show that $5^{5n+1}+4^{5n+2}+3^{5n}$ is a multiple of $11$ when $n=0$.
So $5^{5(0)+1}+4^{5(0)+2}+3^{5(0)}=22$, which is a multiple of $11$. That establishes the base case.
Let $k\geq 0$ be an integer, and assume that $5^{5k+1}+4^{5k+2}+3^{5k}$ is a multiple of 11. 
Then $\exists j\in\ints$ such that $5^{5k+1}+4^{5k+2}+3^{5k}=11j$. This is the inductive hypothesis.

Now, we will show that our hypothesis implies that $5^{5(k+1)+1}+4^{5(k+1)+2}+3^{5(k+1)}=11z$ for some integer $z$. Then

$$5^{5(k+1)+1}+4^{5(k+1)+2}+3^{5(k+1)} = 5^{(5k+1)+5}+4^{(5k+2)+5}+3^{5k+5}$$

\begin{eqnarray*}
    &=& 5^5\cdot 5^{5k+1}+4^5\cdot 4^{5k+2}+3^5\cdot 3^{5k} \\
    \\
    &=& 3125\cdot 5^{5k+1}+1024\cdot 4^{5k+2}+243\cdot 3^{5k} \\
    \\
    &=& 3124\cdot 5^{5k+1}+1023\cdot 4^{5k+2}+242\cdot 3^{5k}+(5^{5k+1}+4^{5k+2}+3^{5k}) \\
    \\
    &=& 3124\cdot 5^{5k+1}+1023\cdot 4^{5k+2}+242\cdot 3^{5k}+11j \hspace{1em} \mbox{(by the inductive hypothesis)} \\
    \\
    &=& 11(248\cdot 5^{5k+1}+93\cdot 4^{5k+2}+22\cdot 3^{5k}+j) \\
    \\
    &=& 11z \mbox{, where } z=248\cdot 5^{5k+1}+93\cdot 4^{5k+2}+22\cdot 3^{5k}+j\\
\end{eqnarray*}
and $z\in\ints$ since it is the sum of products of integers.
Since we have proved the basis step and the inductive step, we conclude that for all integers $n\geq 0$, $5^{5n+1}+4^{5n+2}+3^{5n}$ is a multiple of $11$. 
That completes the proof by induction$\;_{\square}$\\

\newpage

\resume{enumerate}
\item For all positive integers $n$, $\displaystyle \sum_{k=1}^n k(C(n,k))^2=n C(2n-1,n-1)$.\\

Hint:  Consider forming committees of $n$ people (with a chairperson) that are chosen from a group of $n$ Oregonians and $n$ Washingtonians such that the chairperson is an Oregonian.

\end{enumerate}

\bigskip
\bigskip

\emph{Note}: No changes were made since the second draft.

\emph{Proof}: Let $A$ and $B$ be disjoint sets of $n$ distinct elements where $n\geq 1$ is an integer.
The number of ways to form subsets of $n$ elements where at least $1$ element must be from set $A$ is $nC(2n-1,n-1)$ since there are $n$ ways to choose $1$ element from $A$ and $2n-1$ elements remaining to choose $n-1$ elements.

Another way to count this is to count the number that involve $k$ elements from set $A$ and $n-k$ elements from set $B$, and sum over the possible values of $k$, which are $k=1$ to $k=n$.

There are $k$ ways to choose the $1$ required element from set $A$, $C(n, k)$ ways to select $k$ elements from set $A$, and $C(n, n-k)$ ways to select $n-k$ elements from set $B$, for $k=1, 2,..., n$. \\
So the number of ways is $\displaystyle \sum_{k=1}^n kC(n,k)C(n, n-k)$.
We know that $C(n,k)=C(n,n-k)$ because the number of ways to choose $k$ elements from $n$ objects is the same as choosing the $n-k$ elements to be left out.
Therefore, the summation can be written as $\displaystyle \sum_{k=1}^n k(C(n,k))^2$, which we have shown to be equal to $\displaystyle nC(2n-1,n-1)\;_{\square}$ \\

\end{document}
